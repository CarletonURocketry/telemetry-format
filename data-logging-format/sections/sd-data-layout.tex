\subsection{Master Boot Record}

The CU InSpace Avionics system requires SD cards that have been formated with a
Master Boot Record (MBR) in block 0. 

\subsection{Data Partitions}

CU InSpace data is logged into partitions with partion type \textbf{0x89}. A
data logging system will always log into the first valid partition on the disk.
If the first partition becomes full it may continue logging into later
partitions.

CU InSpace logging data partitions are layed out with a single super block
followed by one or more SD Card blocks containing "blocks" of application data.
The application data "blocks" each start with a header which indicates the type
of data in the block and the length of the block. Application data blocks are
stored contiguously and may be stored accross SD Card block boundries.

\subsubsection{Super Block}

Every CU InSpace data partition must start with a Super Block which follows the
format shown in figure \ref{format:superblock}.

\begin{figure}[h]
\centering
\begin{bytefield}[leftcurly=.,leftcurlyspace=0pt,bitwidth=0.03\linewidth]{32}
    \bitheader{0-31} \\
    \begin{leftwordgroup}{0x00}
        \wordbox{2}{Magic Number}
    \end{leftwordgroup} \\
    \begin{leftwordgroup}{0x08}
        \bitbox{8}{version} & \bitbox{1}{\rotatebox{90}{\tiny{Cont.}}} & 
                \bitbox{23}{\color{lightgray}\rule{\width}{\height}} 
    \end{leftwordgroup} \\
    \begin{leftwordgroup}{0x0C}
        \wordbox{1}{Parition Length}
    \end{leftwordgroup} \\

    \wordbox[lrt]{1}{Reserved} \\
    \cskippedwords{lightgray} \\
    \wordbox[lrb]{1}{} \\

    \begin{leftwordgroup}{0x60}
        \wordbox{1}{Flight 0 First Data Block Offset}
    \end{leftwordgroup} \\
    \begin{leftwordgroup}{0x64}
        \wordbox{1}{Flight 0 Last Data Block Offset}
    \end{leftwordgroup} \\
    \begin{leftwordgroup}{0x68}
        \wordbox{1}{Flight 0 Timestamp}
    \end{leftwordgroup} \\
    \begin{leftwordgroup}{0x6C}
        \wordbox{1}{Flight 1 First Data Block Offset}
    \end{leftwordgroup} \\
    \begin{leftwordgroup}{0x70}
        \wordbox{1}{Flight 1 Last Data Block Offset}
    \end{leftwordgroup} \\
    \begin{leftwordgroup}{0x74}
        \wordbox{1}{Flight 1 Timestamp}
    \end{leftwordgroup} \\
   
    \wordbox[lrt]{1}{} \\
    \skippedwords \\
    \wordbox[lrb]{1}{} \\

    \begin{leftwordgroup}{0x1D4}
        \wordbox{1}{Flight 31 First Data Block Offset}
    \end{leftwordgroup} \\
    \begin{leftwordgroup}{0x1D8}
        \wordbox{1}{Flight 31 Last Data Block Offset}
    \end{leftwordgroup} \\
    \begin{leftwordgroup}{0x1DC}
        \wordbox{1}{Flight 31 Timestamp}
    \end{leftwordgroup} \\

    \wordbox[lrt]{1}{Reserved} \\
    \cskippedwords{lightgray} \\
    \wordbox[lrb]{1}{} \\

    \begin{leftwordgroup}{0x1F8}
        \wordbox{2}{Magic Number}
    \end{leftwordgroup} \\
\end{bytefield}
\caption{Super block format}
\label{format:superblock}
\end{figure}

\subsection{Data Blocks}

Every data block starts with a four byte block header. This header indicates the
type of information contained in the block. The block header format is described
in figure \ref{format:block-header}.

\begin{figure}[h]
\centering
\begin{bytefield}[bitwidth=0.03\linewidth]{32}
    \bitheader{0-31} \\
    \bitbox{6}{Class} &
    \bitbox{10}{Type} &
    \bitbox{16}{Length}
\end{bytefield}
\caption{Block header format}
\label{format:block-header}
\end{figure}

The \emph{Class} and \emph{Type} fields are used to specify how the data block
should be interpreted. The \emph{Length} field specifies the length in bytes of
the data block including the block header. This length field can be used to find
the start of the next data block when parsing and can also be used in parsing
data blocks for which the format does not specify a fixed length. The
\emph{Length} value must always be a multiple of four bytes. A length of zero is
invalid.

Table \ref{table:block-classes} shows the possible values for the \emph{Class}
field in the data block header. Section \ref{sec:block-formats} describes the
data formats for blocks of each of the classes.

\begin{table*}[htb]
\centering
\begin{tabular}{@{}ll@{}}
\toprule
Data Block Class    &   Description \\
\midrule
0x0                 &   Logging Metadata (see section \ref{subsec:logging-metadata-blocks}) \\
0x1                 &   Telemetry Data (see section \ref{subsec:telemetry-data-blocks})\\
0x2                 &   Diagnostic Data (see section \ref{subsec:diagnostic-data-blocks}) \\
0x3 through 0x3F    &   Reserved for future use \\
\bottomrule
\end{tabular}
\caption{Data Block Classes}
\label{table:block-classes}
\end{table*}

