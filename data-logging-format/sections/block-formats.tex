\subsection{Logging Metadata Blocks}
\label{subsec:logging-metadata-blocks}

Logging metadata blocks are used to store metadata generated by the logging system and information required in order to
correctly parse the logged data.

The block types within the Logging Metadata class are described in table \ref{table:logging-metadata-types}.

\begin{table*}[htb]
	\centering
	\begin{tabular}{@{}ll@{}}
		\toprule
		Data Type         & Description             \\
		\midrule
		0x0               & Spacer block            \\
		0x1 through 0x3FF & Reserved for future use \\
		\bottomrule
	\end{tabular}
	\caption{Logging metadata block types}
	\label{table:logging-metadata-types}
\end{table*}

\subsubsection{Spacer Block}

Spacer blocks are used when writing telemetry data to indicate a region that does not contain any valid data. When
telemetry is being written one SD card block at a time without the ability to have telemetry data overflow from one
block into another, a spacer block can be used at the end of the SD card block to take up any excess bytes.

When parsing, spacer blocks should always be skipped using the length field in the block header and no attempt should
be made to parse a spacer block's payload.

\subsection{Telemetry Data Blocks}
\label{subsec:telemetry-data-blocks}

Data blocks of the Telemetry class follow the formats specified in the CU InSpace Radio Packet Format document.

\subsection{Diagnostic Data Blocks}
\label{subsec:diagnostic-data-blocks}

Data blocks with the Diagnostic Data class are used to store information which is intended to be used for debugging.

The block types within the Diagnostic Data class are described in table \ref{table:diagnostic-data-types}.

\begin{table*}[htb]
	\centering
	\begin{tabular}{@{}ll@{}}
		\toprule
		Data Type         & Description             \\
		\midrule
		0x0               & Log message             \\
		0x1               & Outgoing radio packet   \\
		0x2               & Incoming radio packet   \\
		0x3 through 0x3FF & Reserved for future use \\
		\bottomrule
	\end{tabular}
	\caption{Diagnostic data block types}
	\label{table:diagnostic-data-types}
\end{table*}

\subsubsection{Log Message}

Log messages are string messages intended to provide human readable debugging output. They are encoded as UTF-8 strings
and do not require a terminating NULL character because the string length can be extrapolated from the block length.
Every log message is preceded by a 32 bit unsigned mission time value.

Note that like all other data blocks log messages must be a multiple of four bytes long. If the string to be logged is
not a multiple of four bytes it must be padded with NULL characters at the end to make it fit the required alignment.
The length recorded in the block header must be a multiple of four, and therefore must include any NULL padding bytes
at the end of the string.
