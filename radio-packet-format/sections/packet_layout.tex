\section{Packet Layout}

CU InSpace's radio packets consist of a packet header followed by one or more segments called blocks. Each block has
its own header followed by a variable amount of data determined by the block type. The layout of a packet is shown in
figure \ref{format:packet}.

\textbf{All fields in this specification are little endian.}

\begin{figure}[H]
    \centering
    \begin{bytefield}{16}
        \wordbox{2}{Packet Header} \\
        \begin{rightwordgroup}{Block 1}
            \wordbox{2}{Block Header} \\
            \wordbox[lrt]{1}{Block Contents} \\
            \skippedwords \\
            \wordbox[lrb]{1}{}
        \end{rightwordgroup} \\
        \wordbox[]{1}{$\vdots$} \\[1ex]
        \begin{rightwordgroup}{Block $N$}
            \wordbox{2}{Block Header} \\
            \wordbox[lrt]{1}{Block Contents} \\
            \skippedwords \\
            \wordbox[lrb]{1}{}
        \end{rightwordgroup} \\
    \end{bytefield}
    \caption{Layout of radio packet}
    \label{format:packet}
\end{figure}

\subsection{Packet Header} \label{sec:pkt-hdr}

Every packet is preceded by a 13 byte header. The header contains an amateur radio call sign and other information which
describes the packet as a whole. The packet header follows the format described in figure \ref{format:packet-header}.

\begin{figure}[H]
    \centering
    \begin{bytefield}[bitwidth=0.03\linewidth]{32}
        \bitheader{0-31} \\
        \wordbox[lrt]{1}{Call Sign} \\ \bitbox[lr]{32}{} \\
        \bitbox[lrb]{8}{} & \bitbox{16}{Timestamp} & \bitbox{8}{Num Blocks} \\
        & \bitbox{8}{Packet Number} \\
    \end{bytefield}
    \caption{Packet header format}
    \label{format:packet-header}
\end{figure}

\paragraph{Call Sign}

The first 9 bytes of the packet must contain an amateur radio call sign in ASCII. If the call sign being used is less
than 9 bytes long, the call sign will be padded with NULL characters at the end.

The call sign field is 9 bytes long because this field is required to accommodate more than just a single Canadian call
sign of maximum 6 characters, but also Canadian call signs being used in the United States. Such call signs must be
suffixed with the call zone in which they are being used. \cite{foreign-broadcast}

For example, the Canadian amateur call sign of \texttt{VA3INI} would become \texttt{VA3INI/W5}, where \texttt{W5} is
the 5th call zone (coincidentally the call zone for the location where Spaceport America Cup is hosted).

There are 10 call zones in the United States, represented using digits 0-9. \cite{us-callzones} The digit is preceded
by a forward slash and a capital 'W', for a total of three additional characters on top of the original 6 for a
Canadian HAM radio call sign.

\paragraph{Timestamp}

This field is an unsigned 16 bit integer representing a number of half-minutes since power on. This timestamp serves as
a base timestamp to which measurement timestamp offsets are added to.

For instance, the packet header may contain a value of 1 in this field, indicating the base timestamp is 30 seconds since
power on. A subsequent block may be a temperature block with a measurement timestamp field containing the value of +12
milliseconds. This value is to be added to the packet header field, meaning that the temperature measurement was taken
at 30 seconds and 12 milliseconds after power on.

This schema allows millisecond resolution on measurement time, while preserving a maximum mission duration of around
546 hours, as seen in equation \ref{eq:max-mission-time}

\begin{equation} \label{eq:max-mission-time}
    \frac{2^{16} - 1}{2*60} = 546.125 \\
\end{equation}

\paragraph{Num Blocks}

This field is an unsigned 8 bit integer that contains the total number of blocks in the packet.

\paragraph{Packet Number}

This field is an unsigned 8 bit integer. It is a rolling counter of the packet sequence number, starting at 0 and
increasing until 255 before rolling over again. Reading this field off of received packets should give an indication of
whether any packets were lost in transmission.

\subsection{Block Header}

Every block starts with a block header. This header indicates the type of information contained in the subsequent
block, which allows for parsing of the received data. The block header is formatted as described in Figure
\ref{format:block-header}.

\begin{figure}[H]
    \centering
    \begin{bytefield}[bitwidth=0.03\linewidth]{8}
        \bitheader{0-7} \\
        \bitbox{8}{Type} &
    \end{bytefield}
    \caption{Block header format}
    \label{format:block-header}
\end{figure}

\paragraph{Type}

The block type field indicates the information contained in the block. Possible values are listed in table
\ref{table:block-types}. Block types are discussed in more detail in section \ref{sec:block-formats}.

\begin{table}[H]
    \centering
    \begin{tabular}{@{}ll@{}}
        \toprule
        Block type                           & Value             \\
        \midrule
        Altitude above sea level             & 0x00              \\
        Altitude above launch level          & 0x01              \\
        Temperature                          & 0x02              \\
        Pressure                             & 0x03              \\
        Linear acceleration                  & 0x04              \\
        Angular velocity                     & 0x06              \\
        Humidity                             & 0x07              \\
        Coordinates (latitude and longitude) & 0x08              \\
        Voltage                              & 0x09              \\
        Reserved for future use              & 0x0A through 0xFF \\
        \bottomrule
    \end{tabular}
    \caption{Block types}
    \label{table:block-types}
\end{table}
