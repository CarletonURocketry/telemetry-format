\subsubsection{Voltage}

The voltage data block reports the voltage measured on a specific electrical trace within the rocket. The measurement
is associated with a generic numerical ID for identification by the receiver. This allows multiple voltage points to be
measured within the rocket. The format of the voltage block payload is described in figure \ref{format:telem-voltage}.

\begin{figure}[H]
    \centering
    \begin{bytefield}[bitwidth=0.03\linewidth]{16}
        \bitheader{0-15} \\
        \wordbox{1}{Measurement Time} \\
        \wordbox{1}{Voltage} \\
        \bitbox{8}{ID} \\
    \end{bytefield}
    \caption{Voltage data block format}
    \label{format:telem-voltage}
\end{figure}

\blocktimestampexp

\paragraph{Voltage}

The measured voltage in units of millivolts. This field is a signed 16 bit integer.

\paragraph{ID}

A numerical identifier associated with the voltage measurement for identification by the receiver. This field is an
unsigned 8 bit integer, allowing for up to 255 unique identifiers.
