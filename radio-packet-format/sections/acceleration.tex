\subsection{Linear Acceleration}

The linear acceleration block is used to convey generic 3-axis accelerometer data. This block is intended to abstract
over the details of any individual accelerometer. The format of the acceleration block payload is described in figure
\ref{format:telem-acceleration}.

\begin{figure}[H]
    \centering
    \begin{bytefield}[bitwidth=0.03\linewidth]{16}
        \bitheader{0-15} \\
        \wordbox{1}{Measurement Time} \\
        \wordbox{1}{X-Axis} \\
        \wordbox{1}{Y-Axis} \\
        \wordbox{1}{Z-Axis} \\
    \end{bytefield}
    \caption{Acceleration Data Payload Format}
    \label{format:telem-acceleration}
\end{figure}

\blocktimestampexp

\paragraph{*-Axis}

These fields represent the acceleration measurements for each axis. These fields are signed integers in two's
complement format.

The unit of measurement for each axis is measured in centimetres per second squared. That is, dividing the value in
each axis field by 100 will give the standard acceleration unit of metres per second squared.

The acceleration measurements are relative to the rocket's heading. If the IMU Z-axis is in parallel with the rocket,
then the acceleration in the z-axis is also a parallel vector.
