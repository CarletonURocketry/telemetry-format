\subsubsection{Linear Acceleration}

The linear acceleration block is used to convey generic 3-axis accelerometer data. This block is intended to abstract
over the details of any individual accelerometer. The format of the acceleration block payload is described in figure
\ref{format:telem-acceleration}.

\begin{figure}[H]
    \centering
    \begin{bytefield}[bitwidth=0.03\linewidth]{16}
        \bitheader{0-15} \\
        \wordbox{1}{Measurement Time} \\
        \wordbox{1}{X-Axis} \\
        \wordbox{1}{Y-Axis} \\
        \wordbox{1}{Z-Axis} \\
    \end{bytefield}
    \caption{Acceleration Data Payload Format}
    \label{format:telem-acceleration}
\end{figure}

\blocktimestampexp

\paragraph{*-Axis}

These fields represent the acceleration measurements for each axis. These fields are signed integers in two's
complement format.

The unit of measurement for each axis is measured in centimetres per second squared. That is, dividing the value in
each axis field by 100 will give the standard acceleration unit of metres per second squared.

If the sub-type specified in the block header is linear acceleration relative to the rocket heading, the data in these
fields should be interpreted as having heading's relative to the rocket's current position in 3D space. This means that
measurements in the Z-axis may not necessarily provide vertical acceleration; the measurement could be horizontal if
the rocket is flying directly horizontally.

If the sub-type specified in the block header is linear acceleration relative to the ground, the data in these fields
can be interpreted as never changing in direction. Z-axis acceleration can be interpreted as vertical acceleration in
this case.

All possible sub-types of linear acceleration are listed in Table \ref{table:data-subtypes}.
