\subsubsection{Request Telemetry Data}

This command is used to explicitly request that the rocket avionics sent particular telemetry data back. Up to four
different telemetry data blocks can be requested per command. The payload data for the request telemetry data command
is made up of four bytes, each of which may indicate a requested data block sub-type. The format of the request block
payload is described in figure \ref{format:req-telem-data}.

\begin{figure}[H]
    \centering
    \begin{bytefield}[bitwidth=0.03\linewidth]{32}
        \bitheader{0-31} \\
        \bitbox{6}{Data Sub-type} &
        \bitbox{1}{\color{lightgray}\rule{\width}{\height}} & \bitbox{1}{U} &
        \bitbox{6}{Data Sub-type} &
        \bitbox{1}{\color{lightgray}\rule{\width}{\height}} & \bitbox{1}{U} &
        \bitbox{6}{Data Sub-type} &
        \bitbox{1}{\color{lightgray}\rule{\width}{\height}} & \bitbox{1}{U} &
        \bitbox{6}{Data Sub-type} &
        \bitbox{1}{\color{lightgray}\rule{\width}{\height}} & \bitbox{1}{U}
    \end{bytefield}
    \caption{Telemetry Data Request Payload Format}
    \label{format:req-telem-data}
\end{figure}

\paragraph{Data Sub-type}

The sub-type of data block that is being requested.

\paragraph{Used (U)}

For each of the four bytes in the request telemetry packet’s payload this bit indicates whether the byte contains a
valid request. Any bytes for which this bit is not set should be ignored.
