\subsection{Command Blocks}

Command blocks contain instructions or requests for information. The possible sub-types for command blocks are listed
in table \ref{table:command-subtypes}.

\begin{table*}[htb]
    \centering
    \begin{tabular}{@{}ll@{}}
        \toprule
        Value            & Description             \\
        \midrule
        0x0              & Reset rocket avionics   \\
        0x1              & Request telemetry data  \\
        0x2              & Deploy parachute        \\
        0x3              & Tare sensors            \\
        0x4 through 0x3F & Reserved for future use \\
        \bottomrule
    \end{tabular}
    \caption{Command Block Sub-types}
    \label{table:command-subtypes}
\end{table*}

\subsubsection{Reset Rocket Avionics}
When this block is received by the rocket avionics the rocket’s microcontroller will be reset. If this block’s ACK
requested bit is set the acknowledgement will be sent before the avionics reset. The acknowledgements for any other
blocks in the same packet as the reset command are not guaranteed to be sent. This block sub-type does not have any
payload data.

\subsubsection{Request Telemetry Data}
This command is used to explicitly request that the rocket avionics sent particular telemetry data back. Up to four
different telemetry data blocks can be requested per command. The payload data for the request telemetry data command
is made up of four bytes, each of which may indicate a requested data block sub-type. The format of the request block
payload is described in figure \ref{format:req-telem-data}.

\begin{figure}[h]
    \centering
    \begin{bytefield}[bitwidth=0.03\linewidth]{32}
        \bitheader{0-31} \\
        \bitbox{6}{Data Sub-type} &
        \bitbox{1}{\color{lightgray}\rule{\width}{\height}} & \bitbox{1}{U} &
        \bitbox{6}{Data Sub-type} &
        \bitbox{1}{\color{lightgray}\rule{\width}{\height}} & \bitbox{1}{U} &
        \bitbox{6}{Data Sub-type} &
        \bitbox{1}{\color{lightgray}\rule{\width}{\height}} & \bitbox{1}{U} &
        \bitbox{6}{Data Sub-type} &
        \bitbox{1}{\color{lightgray}\rule{\width}{\height}} & \bitbox{1}{U}
    \end{bytefield}
    \caption{Telemetry Data Request Payload Format}
    \label{format:req-telem-data}
\end{figure}

\paragraph{Data Sub-type}
The sub-type of data block that is being requested.

\paragraph{Used (U)}
For each of the four bytes in the request telemetry packet’s payload this bit indicates whether the byte contains a
valid request. Any bytes for which this bit is not set should be ignored.

\subsubsection{Deploy parachute}
This command indicates to the rocket avionics that it should immediately deploy the drogue parachute. This block
sub-type does not have any payload data.

\subsubsection{Tare sensors}
This command tells the rocket to zero out all of its sensors. This command should only be send when the rocket is on
the pad ready to launch. The rocket avionics can assume that when it receives this command it is stationary and in the
correct orientation for launch.
