\subsubsection{Altitude}

The altitude block is used to convey the altitude of the rocket. The format of the altitude block payload is described
in figure \ref{format:telem-altitude}.

\begin{figure}[H]
    \centering
    \begin{bytefield}[bitwidth=0.03\linewidth]{32}
        \bitheader{0-31} \\
        \wordbox{1}{Measurement Time} \\
        \wordbox{1}{Altitude}
    \end{bytefield}
    \caption{Altitude Data Payload Format}
    \label{format:telem-altitude}
\end{figure}

\paragraph{Measurement Time}
The mission time when the measurement was taken.

\paragraph{Altitude}
The calculated altitude in units of millimetres. This field is a signed 32 bit integer in two's complement format.

If the block header specifies the sub-type for altitude above sea level, this measurement should be interpreted as
millimetres above sea level.

If the block header specifies the sub-type for altitude above launch height, this measurement should be interpreted as
millimetres from the starting height of the rocket.

See Table \ref{table:data-subtypes} for the sub-types altitude can specify.
