\subsubsection{Acceleration}

The acceleration block is used to convey generic 3-axis accelerometer data. This block is intended to abstract over the
details of any individual accelerometer. The format of the acceleration block payload is described in figure
\ref{format:telem-acceleration}.

\begin{figure}[H]
    \centering
    \begin{bytefield}[bitwidth=0.03\linewidth]{32}
        \bitheader{0-31} \\
        \wordbox{1}{Measurement Time} \\
        \bitbox{8}{Full Scale Range} &
        \bitbox{8}{\color{lightgray}\rule{\width}{\height}} &
        \bitbox{16}{X-Axis} \\
        \bitbox{16}{Y-Axis} & \bitbox{16}{Z-Axis}
    \end{bytefield}
    \caption{Acceleration Data Payload Format}
    \label{format:telem-acceleration}
\end{figure}

\paragraph{Measurement Time}
The mission time when the measurement was taken.

\paragraph{Full Scale Range}
The full scale range of the accelerometer in g. This value represents the maximum and minimum acceleration that can be
measured, for example a value of 16 in this field represents a full scale range of \SI{\pm 16}{g}. If the resolution of
the accelerometer is less than 16 bits the full scale range must be adjusted to compensate, for example a 10 bit
accelerometer with a \SI{\pm 2}{g} full scale range would use the value 64 for this field.

This value can be used to calculate the accelerometer sensitivity, which in turn can be used to convert the
acceleration measurements for each axis into useful units. For a measurement $m$ with a full scale range $f$ the
acceleration in g can be found like this:

$$
    \text{acceleration} = m \cdot \frac{f}{2^{15}}
$$

\paragraph{*-Axis}
These fields represent the acceleration measurements for each axis. These fields are signed integers in two's
complement format. If the resolution of the accelerometer is less than 16 bits the values must be sign extended.
