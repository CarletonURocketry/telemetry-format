\subsubsection{KX134-1211 Accelerometer Data}

The format of the KX134-1211 accelerometer data block is described in figure \ref{format:telem-kx134-accel}. This block
can contain a variable amount of acceleration data. This data will be in sets of three measurements (x, y, z), each
measurement is either 8 or 16 bits long and is encoded in two's complement format.

\begin{figure}[h]
    \centering
    \begin{bytefield}[bitwidth=0.03\linewidth]{32}
        \bitheader{0-31} \\
        \wordbox{1}{Measurement Time} \\
        \bitbox{4}{ODR} & \bitbox{2}{\tiny Range} &
        \bitbox{1}{\rotatebox{90}{\tiny Roll}} &
        \bitbox{1}{\rotatebox{90}{\tiny Res}} &
        \bitbox{6}{\color{lightgray}\rule{\width}{\height}} &
        \bitbox{2}{\small Pad} &
        \bitbox[lrt]{16}{} \\
        \wordbox[lr]{1}{Acceleration Data} \\
        \skippedwords \\
        \wordbox[lrb]{1}{} \\
    \end{bytefield}
    \caption{KX134-1211 Accelerometer Data Payload Format}
    \label{format:telem-kx134-accel}
\end{figure}

\paragraph{Measurement Time}
The time of the last measurement in the block.

\paragraph{ODR}
This field encodes the output data rate at which the accelerometer is configured. The possible values for this field
are shown in table \ref{table:kx134-odr}.

\begin{table*}[htb]
    \centering
    \begin{tabular}{@{}ll@{}}
        \toprule
        ODR Value & Output Data Rate     \\
        \midrule
        0x0       & {\SI{0.781}{\hertz}} \\
        0x1       & {\SI{1.563}{\hertz}} \\
        0x2       & {\SI{3.125}{\hertz}} \\
        0x3       & {\SI{6.25}{\hertz}}  \\
        0x4       & {\SI{12.5}{\hertz}}  \\
        0x5       & {\SI{25}{\hertz}}    \\
        0x6       & {\SI{50}{\hertz}}    \\
        0x7       & {\SI{100}{\hertz}}   \\
        0x8       & {\SI{200}{\hertz}}   \\
        0x9       & {\SI{400}{\hertz}}   \\
        0xa       & {\SI{800}{\hertz}}   \\
        0xb       & {\SI{1600}{\hertz}}  \\
        0xc       & {\SI{3200}{\hertz}}  \\
        0xd       & {\SI{6400}{\hertz}}  \\
        0xe       & {\SI{12800}{\hertz}} \\
        0xf       & {\SI{25600}{\hertz}} \\
        \bottomrule
    \end{tabular}
    \caption{KX134-1211 Output Data Rate Values}
    \label{table:kx134-odr}
\end{table*}

\paragraph{Range}
This field encodes the configured full scale range of the accelerometer. The possible values for this field are shown
in table \ref{table:kx134-fsr}.

\begin{table*}[htb]
    \centering
    \begin{tabular}{@{}ll@{}}
        \toprule
        Range Value & Full Scale Range \\
        \midrule
        0x0         & {\SI{\pm 8}{g}}  \\
        0x1         & {\SI{\pm 16}{g}} \\
        0x2         & {\SI{\pm 32}{g}} \\
        0x3         & {\SI{\pm 64}{g}} \\
        \bottomrule
    \end{tabular}
    \caption{KX134-1211 Full Scale Range Values}
    \label{table:kx134-fsr}
\end{table*}

\paragraph{Roll}
This field encode the accelerometer's configured low pass filter roll-off. The possible values for this field are shown
in table \ref{table:kx134-rolloff}.

\begin{table*}[htb]
    \centering
    \begin{tabular}{@{}ll@{}}
        \toprule
        Roll Value & Low Pass Filter Corner Frequency \\
        \midrule
        0x0        & $\text{ODR} / 9$                 \\
        0x1        & $\text{ODR} / 2$                 \\
        \bottomrule
    \end{tabular}
    \caption{KX134-1211 Low Pass Filter Rolloff Values}
    \label{table:kx134-rolloff}
\end{table*}

\paragraph{Res}
This field encodes the output resolution for the accelerometer. The possible values for this field are shown in table
\ref{table:kx134-res}.

\begin{table*}[htb]
    \centering
    \begin{tabular}{@{}ll@{}}
        \toprule
        Res Value & Resolution     \\
        \midrule
        0x0       & {\SI{8}{bit}}  \\
        0x1       & {\SI{16}{bit}} \\
        \bottomrule
    \end{tabular}
    \caption{KX134-1211 Resolution Values}
    \label{table:kx134-res}
\end{table*}

\paragraph{Pad}
This field indicates the number of padding bytes which present after the end of the acceleration data. This field is
important when parsing 8 bit data because it could appear that there is an extra sample due to padding.
