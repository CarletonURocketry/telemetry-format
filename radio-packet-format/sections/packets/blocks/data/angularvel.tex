\subsubsection{Angular Velocity}

The angular velocity block is used to convey generic 3-axis gyroscope data. This block is intended to abstract over the
details of any individual gyroscope. The format of the acceleration block payload is described in figure
\ref{format:telem-angular-velocity}.

\begin{figure}[H]
    \centering
    \begin{bytefield}[bitwidth=0.03\linewidth]{32}
        \bitheader{0-31} \\
        \wordbox{1}{Measurement Time} \\
        \bitbox{16}{Full Scale Range} & \bitbox{16}{X-Axis} \\
        \bitbox{16}{Y-Axis} & \bitbox{16}{Z-Axis}
    \end{bytefield}
    \caption{Angular Velocity Data Payload Format}
    \label{format:telem-angular-velocity}
\end{figure}

\paragraph{Measurement Time}
The mission time when the measurement was taken.

\paragraph{Full Scale Range}
The full scale range of the gyroscope in degrees per second. This value represents the maximum and minimum angular
velocity that can be measured, for example a value of 2000 in this field represents a full scale range of \SI{\pm
    2000}{degrees per second}. If the resolution of the gyroscope is less than 16 bits the full scale range must be
adjusted to compensate, for example a 12 bit gyroscope with a \SI{\pm 500}{degree per second} full scale range would
use the value 4000 for this field.

This value can be used to calculate the gyroscope sensitivity, which in turn can be used to convert the acceleration
measurements for each axis into useful units. For a measurement $m$ with a full scale range $f$ the angular velocity in
degrees per second can be found like this:

$$
    \text{angular velocity} = m \cdot \frac{f}{2^{15}}
$$

\paragraph{*-Axis}
These fields represent the angular velocity measurements for each axis. These fields are signed integers in two's
complement format.
