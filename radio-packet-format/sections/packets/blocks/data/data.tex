\subsection{Data Blocks}

Data blocks contain telemetry data. The possible sub-types data blocks are listed in table \ref{table:data-subtypes}.

Some sub-types may have identical encoding, but a different interpretation. For example, all altitude measurements are
encoded in millimetres. However, "sea level altitude" is measured in millimetres above sea level whilst "launch level
altitude" is measured in millimetres above launch height. When reviewing the encoding for a data type, the description
will mention how the encoded value should be interpreted. For example, in the case of altitude, the millimetre
measurement should be interpreted as being relative to sea level if the block header specifies a sub-type of "sea level
altitude".

Note that all \textbf{mission time} fields are measured in milliseconds since launch, and are all unsigned 32-bit
integers.

Any integer fields larger than 1 byte are in little-endian format. Any signed integers are in two's complement format.
This is because fields are packaged using C structs on a little endian machine (the de-facto standard).

\begin{table}[H]
    \centering
    \begin{tabular}{@{}ll@{}}
        \toprule
        Value             & Description                                    \\
        \midrule
        0x00              & Debug message                                  \\
        0x01              & Altitude above sea level                       \\
        0x02              & Altitude above launch level                    \\
        0x03              & Temperature                                    \\
        0x04              & Pressure                                       \\
        0x05              & Linear acceleration relative to rocket heading \\
        0x06              & Linear acceleration relative to ground         \\
        0x07              & Angular velocity                               \\
        0x08              & Humidity                                       \\
        0x09              & Coordinates (latitude and longitude)           \\
        0x0A              & Voltage                                        \\
        0x0B through 0xFF & Reserved for future use                        \\
        \bottomrule
    \end{tabular}
    \caption{Data Block Subtypes}
    \label{table:data-subtypes}
\end{table}

% DATA BLOCK TYPES %
\subsubsection{Debug Message}

Debug messages are string messages intended to provide human readable debugging output. They are encoded as UTF-8
strings and do not require a terminating NULL character because the string length can be extrapolated from the block
length. Every debug message is preceded by a 32 bit unsigned mission time value.

Note that like all other data blocks debug messages must be a multiple of four bytes long. If the string to be logged
is not a multiple of four bytes it must be padded with NULL characters at the end to make it fit the required
alignment. The length recorded in the block header must be a multiple of four, and therefore must include any NULL
padding bytes at the end of the string.

\subsubsection{Altitude}

The altitude block is used to convey the altitude of the rocket. The format of the altitude block payload is described
in figure \ref{format:telem-altitude}.

\begin{figure}[H]
    \centering
    \begin{bytefield}[bitwidth=0.03\linewidth]{32}
        \bitheader{0-31} \\
        \wordbox{1}{Measurement Time} \\
        \wordbox{1}{Altitude}
    \end{bytefield}
    \caption{Altitude Data Payload Format}
    \label{format:telem-altitude}
\end{figure}

\paragraph{Measurement Time}
The mission time when the measurement was taken.

\paragraph{Altitude}
The calculated altitude in units of \SI{1}{\milli\meter\per LSB}. This field is a signed 32 bit integer in two's
complement format.

\subsubsection{Temperature}

The temperature data block reports the temperature detected from within the rocket in millidegrees Celsius. The format
of the temperature block payload is described in figure \ref{format:telem-temperature}.

\begin{figure}[H]
    \centering
    \begin{bytefield}[bitwidth=0.03\linewidth]{32}
        \bitheader{0-31} \\
        \wordbox{1}{Measurement Time} \\
        \wordbox{1}{Temperature} \\
    \end{bytefield}
    \caption{Temperature data block format}
    \label{format:telem-temperature}
\end{figure}

\paragraph{Measurement Time}
The mission time when the measurement was taken.

\paragraph{Temperature}
The measured temperature in units of millidegrees Celsius. This field is a signed 32 bit integer in two's complement
format.

\subsubsection{Pressure}

The pressure data block reports the pressure detected from within the rocket in Pascals. The format of the pressure
block payload is described in figure \ref{format:telem-pressure}.

\begin{figure}[H]
    \centering
    \begin{bytefield}[bitwidth=0.03\linewidth]{32}
        \bitheader{0-31} \\
        \wordbox{1}{Measurement Time} \\
        \wordbox{1}{Pressure} \\
    \end{bytefield}
    \caption{Pressure data block format}
    \label{format:telem-pressure}
\end{figure}

\paragraph{Measurement Time}
The mission time when the measurement was taken.

\paragraph{Pressure}
The measured pressure in units of Pascals. This field is an unsigned 32 bit integer.

\subsection{Linear Acceleration}

The linear acceleration block is used to convey generic 3-axis accelerometer data. This block is intended to abstract
over the details of any individual accelerometer. The format of the acceleration block payload is described in figure
\ref{format:telem-acceleration}.

\begin{figure}[H]
    \centering
    \begin{bytefield}[bitwidth=0.03\linewidth]{16}
        \bitheader{0-15} \\
        \wordbox{1}{Measurement Time} \\
        \wordbox{1}{X-Axis} \\
        \wordbox{1}{Y-Axis} \\
        \wordbox{1}{Z-Axis} \\
    \end{bytefield}
    \caption{Acceleration Data Payload Format}
    \label{format:telem-acceleration}
\end{figure}

\blocktimestampexp

\paragraph{*-Axis}

These fields represent the acceleration measurements for each axis. These fields are signed integers in two's
complement format.

The unit of measurement for each axis is measured in centimetres per second squared. That is, dividing the value in
each axis field by 100 will give the standard acceleration unit of metres per second squared.

The acceleration measurements are relative to the rocket's heading. If the IMU Z-axis is in parallel with the rocket,
then the acceleration in the z-axis is also a parallel vector.

\subsubsection{Angular Velocity}

The angular velocity block is used to convey generic 3-axis gyroscope data. This block is intended to abstract over the
details of any individual gyroscope. The format of the acceleration block payload is described in figure
\ref{format:telem-angular-velocity}.

\begin{figure}[h]
    \centering
    \begin{bytefield}[bitwidth=0.03\linewidth]{32}
        \bitheader{0-31} \\
        \wordbox{1}{Measurement Time} \\
        \bitbox{16}{Full Scale Range} & \bitbox{16}{X-Axis} \\
        \bitbox{16}{Y-Axis} & \bitbox{16}{Z-Axis}
    \end{bytefield}
    \caption{Angular Velocity Data Payload Format}
    \label{format:telem-angular-velocity}
\end{figure}

\paragraph{Measurement Time}
The mission time when the measurement was taken.

\paragraph{Full Scale Range}
The full scale range of the gyroscope in degrees per second. This value represents the maximum and minimum angular
velocity that can be measured, for example a value of 2000 in this field represents a full scale range of \SI{\pm
    2000}{degrees per second}. If the resolution of the gyroscope is less than 16 bits the full scale range must be
adjusted to compensate, for example a 12 bit gyroscope with a \SI{\pm 500}{degree per second} full scale range would
use the value 4000 for this field.

This value can be used to calculate the gyroscope sensitivity, which in turn can be used to convert the acceleration
measurements for each axis into useful units. For a measurement $m$ with a full scale range $f$ the angular velocity in
degrees per second can be found like this:

$$
    \text{angular velocity} = m \cdot \frac{f}{2^{15}}
$$

\paragraph{*-Axis}
These fields represent the angular velocity measurements for each axis. These fields are signed integers in two's
complement format.

\subsection{Humidity}

The humidity block is used to convey the relative humidity inside of the rocket. The format of the humidity block is
described in figure \ref{format:telem-humidity}.

\begin{figure}[H]
    \centering
    \begin{bytefield}[bitwidth=0.03\linewidth]{16}
        \bitheader{0-15} \\
        \wordbox{1}{Measurement Time} \\
        \wordbox{2}{Humidity}
    \end{bytefield}
    \caption{Humidity Data Payload Format}
    \label{format:telem-humidity}
\end{figure}

\blocktimestampexp

\paragraph{Humidity}

The calculated relative humidity in ten thousandths of a percent (i.e 1\% is 100). This field is an unsigned 32 bit
integer.

\subsubsection{Coordinates}

The coordinates block is used to convey the geographical (latitude and longitude) coordinates of the rocket. The format
of the coordinates block is described in figure \ref{format:telem-coordinate}.

\begin{figure}[H]
    \centering
    \begin{bytefield}[bitwidth=0.03\linewidth]{32}
        \bitheader{0-31} \\
        \wordbox{1}{Measurement Time} \\
        \wordbox{1}{Latitude} \\
        \wordbox{1}{Longitude}
    \end{bytefield}
    \caption{Coordinate Data Payload Format}
    \label{format:telem-coordinate}
\end{figure}

\paragraph{Measurement Time}
The mission time when the measurement was taken.

\paragraph{Latitude}
Measured in micro-degrees. This field is a signed 32 bit integer. To convert to degrees, divide by $10^7$.

\paragraph{Longitude}
Measured in micro-degrees. This field is a signed 32 bit integer. To convert to degrees, divide by $10^7$.

\subsubsection{Voltage}

The voltage data block reports the voltage measured on a specific electrical trace within the rocket. The measurement
is associated with a generic numerical ID for identification by the receiver. This allows multiple voltage points to be
measured within the rocket. The format of the voltage block payload is described in figure \ref{format:telem-voltage}.

\begin{figure}[H]
    \centering
    \begin{bytefield}[bitwidth=0.03\linewidth]{16}
        \bitheader{0-15} \\
        \wordbox{1}{Measurement Time} \\
        \wordbox{1}{Voltage} \\
        \bitbox{8}{ID} \\
    \end{bytefield}
    \caption{Voltage data block format}
    \label{format:telem-voltage}
\end{figure}

\blocktimestampexp

\paragraph{Voltage}

The measured voltage in units of millivolts. This field is a signed 16 bit integer.

\paragraph{ID}

A numerical identifier associated with the voltage measurement for identification by the receiver. This field is an
unsigned 8 bit integer, allowing for up to 255 unique identifiers.

