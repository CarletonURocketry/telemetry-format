\subsection{Data Blocks}

Data blocks contain telemetry data. The possible sub-types data blocks are listed in table \ref{table:data-subtypes}.

\begin{table*}[htb]
    \centering
    \begin{tabular}{@{}ll@{}}
        \toprule
        Value             & Description                   \\
        \midrule
        0x00              & Debug message                 \\
        0x01              & Status                        \\
        0x02              & Startup message               \\
        0x03              & Altitude                      \\
        0x04              & Acceleration                  \\
        0x05              & Angular velocity              \\
        0x06              & GNSS location                 \\
        0x07              & GNSS metadata                 \\
        0x08              & Power information             \\
        0x09              & Temperatures                  \\
        0x0A              & MPU-9250 IMU data             \\
        0x0B              & KX134-1211 accelerometer data \\
        0x0C through 0x3F & Reserved for future use       \\
        \bottomrule
    \end{tabular}
    \caption{Data Block Subtypes}
    \label{table:data-subtypes}
\end{table*}

\subsubsection{Debug Message}

Debug messages are string messages intended to provide human readable debugging output. They are encoded as UTF-8
strings and do not require a terminating NULL character because the string length can be extrapolated from the block
length. Every debug message is preceded by a 32 bit unsigned mission time value.

Note that like all other data blocks debug messages must be a multiple of four bytes long. If the string to be logged
is not a multiple of four bytes it must be padded with NULL characters at the end to make it fit the required
alignment. The length recorded in the block header must be a multiple of four, and therefore must include any NULL
padding bytes at the end of the string.

\subsubsection{Status}

The status block describes the state of the various software components of the avionics system.

\begin{figure}[h]
    \centering
    \begin{bytefield}[bitwidth=0.03\linewidth]{32}
        \bitheader{0-31} \\
        \wordbox{1}{Mission Time} \\
        \bitbox{16}{\color{lightgray}\rule{\width}{\height}} &
        \bitbox{3}{\scriptsize{KX134 State}} &
        \bitbox{3}{\scriptsize{Alt State}} &
        \bitbox{3}{\scriptsize{IMU State}} &
        \bitbox{3}{\scriptsize{SD State}} &
        \bitbox{4}{\scriptsize{Deployment State}} \\
        \wordbox{1}{SD Blocks Recorded} \\
        \wordbox{1}{SD Checkouts Missed} \\
    \end{bytefield}
    \caption{Status Data Payload Format}
    \label{format:telem-status}
\end{figure}

\paragraph{Measurement Time}
The mission time at which the states where recorded.

\begin{table*}[htb]
    \centering
    \begin{tabular}{@{}ll@{}}
        \toprule
        Sensor State Value & Description               \\
        \midrule
        0x0                & None (sensor not enabled) \\
        0x1                & Initializing              \\
        0x2                & Running                   \\
        0x3                & Self test failed          \\
        0x4                & Failed                    \\
        0x5 through 0x7    & Reserved for future use   \\
        \bottomrule
    \end{tabular}
    \caption{Sensor States}
    \label{table:sensor-state}
\end{table*}

\paragraph{KX134 State}
The current state of the KX134-1211 accelerometer (if enabled). This field will contain one of the values from table
\ref{table:sensor-state}.

\paragraph{Alt State}
The current state of the altimeter (if enabled). This field will contain one of the values from table
\ref{table:sensor-state}.

\paragraph{IMU State}
The current state of the inertial measurement unit (if enabled). This field will contain one of the values from table
\ref{table:sensor-state}.

\begin{table*}[htb]
    \centering
    \begin{tabular}{@{}ll@{}}
        \toprule
        SD Card Driver State Value & Description             \\
        \midrule
        0x0                        & Card not present        \\
        0x1                        & Initializing            \\
        0x2                        & Read                    \\
        0x3                        & Failed                  \\
        0x4 through 0x7            & Reserved for future use \\
        \bottomrule
    \end{tabular}
    \caption{SD Card Driver States}
    \label{table:sd-state}
\end{table*}

\paragraph{SD State}
The current state of the SD card driver. This field will contain one of the values from table \ref{table:sd-state}.

\begin{table*}[htb]
    \centering
    \begin{tabular}{@{}ll@{}}
        \toprule
        Deployment Service State Value & Description             \\
        \midrule
        0x0                            & Idle                    \\
        0x1                            & Armed                   \\
        0x2                            & Powered ascent          \\
        0x3                            & Coasting ascent         \\
        0x4                            & Deployment in progress  \\
        0x5                            & Descent                 \\
        0x6                            & Recovery                \\
        0x7 through 0xF                & Reserved for future use \\
        \bottomrule
    \end{tabular}
    \caption{Deployment Service States}
    \label{table:deployment-state}
\end{table*}

\paragraph{Deployment State}
The current state of the deployment service state machine. This field will contain one of the values from table
\ref{table:deployment-state}.

\paragraph{SD Blocks Recorded}
The number of blocks of telemetry data that have been recorded to the SD card during the current flight. Not this this
refers to storage device blocks (512 byte units) rather than telemetry format blocks.

\paragraph{SD Checkouts Missed}
The number of telemetry format data blocks that have been dropped because there was not enough buffer space free to log
them to the SD card.

\subsubsection{Altitude}

The altitude block is used to convey altimeter data. This block contains both the altitude value calculated by the
rocket and the raw pressure and temperature values if they are available. The format of the altitude block payload is
described in figure \ref{format:telem-altitude}.

\begin{figure}[h]
    \centering
    \begin{bytefield}[bitwidth=0.03\linewidth]{32}
        \bitheader{0-31} \\
        \wordbox{1}{Measurement Time} \\
        \wordbox{1}{Pressure} \\
        \wordbox{1}{Temperature} \\
        \wordbox{1}{Altitude}
    \end{bytefield}
    \caption{Altitude Data Payload Format}
    \label{format:telem-altitude}
\end{figure}

\paragraph{Measurement Time}
The mission time when the measurement was taken.

\paragraph{Pressure}
The measured pressure in units of Pascals. This field is a signed 32 bit integer in two's complement format.

\paragraph{Temperature}
The measured temperature in units of 1 millidegree Celsius/LSB. This field is a signed 32 bit integer in two's
complement format.

\paragraph{Altitude}
The calculated altitude in units of \SI{1}{\milli\meter\per LSB}. This field is a signed 32 bit integer in two's
complement format.

\subsubsection{Acceleration}

The acceleration block is used to convey generic 3-axis accelerometer data. This block is intended to abstract over the
details of any individual accelerometer. The format of the acceleration block payload is described in figure
\ref{format:telem-acceleration}.

\begin{figure}[h]
    \centering
    \begin{bytefield}[bitwidth=0.03\linewidth]{32}
        \bitheader{0-31} \\
        \wordbox{1}{Measurement Time} \\
        \bitbox{8}{Full Scale Range} &
        \bitbox{8}{\color{lightgray}\rule{\width}{\height}} &
        \bitbox{16}{X-Axis} \\
        \bitbox{16}{Y-Axis} & \bitbox{16}{Z-Axis}
    \end{bytefield}
    \caption{Acceleration Data Payload Format}
    \label{format:telem-acceleration}
\end{figure}

\paragraph{Measurement Time}
The mission time when the measurement was taken.

\paragraph{Full Scale Range}
The full scale range of the accelerometer in g. This value represents the maximum and minimum acceleration that can be
measured, for example a value of 16 in this field represents a full scale range of \SI{\pm 16}{g}. If the resolution of
the accelerometer is less than 16 bits the full scale range must be adjusted to compensate, for example a 10 bit
accelerometer with a \SI{\pm 2}{g} full scale range would use the value 64 for this field.

This value can be used to calculate the accelerometer sensitivity, which in turn can be used to convert the
acceleration measurements for each axis into useful units. For a measurement $m$ with a full scale range $f$ the
acceleration in g can be found like this:

$$
    \text{acceleration} = m \cdot \frac{f}{2^{15}}
$$

\paragraph{*-Axis}
These fields represent the acceleration measurements for each axis. These fields are signed integers in two's
complement format. If the resolution of the accelerometer is less than 16 bits the values must be sign extended.

\subsubsection{Angular Velocity}

The angular velocity block is used to convey generic 3-axis gyroscope data. This block is intended to abstract over the
details of any individual gyroscope. The format of the acceleration block payload is described in figure
\ref{format:telem-angular-velocity}.

\begin{figure}[h]
    \centering
    \begin{bytefield}[bitwidth=0.03\linewidth]{32}
        \bitheader{0-31} \\
        \wordbox{1}{Measurement Time} \\
        \bitbox{16}{Full Scale Range} & \bitbox{16}{X-Axis} \\
        \bitbox{16}{Y-Axis} & \bitbox{16}{Z-Axis}
    \end{bytefield}
    \caption{Angular Velocity Data Payload Format}
    \label{format:telem-angular-velocity}
\end{figure}

\paragraph{Measurement Time}
The mission time when the measurement was taken.

\paragraph{Full Scale Range}
The full scale range of the gyroscope in degrees per second. This value represents the maximum and minimum angular
velocity that can be measured, for example a value of 2000 in this field represents a full scale range of \SI{\pm
    2000}{degrees per second}. If the resolution of the gyroscope is less than 16 bits the full scale range must be
adjusted to compensate, for example a 12 bit gyroscope with a \SI{\pm 500}{degree per second} full scale range would
use the value 4000 for this field.

This value can be used to calculate the gyroscope sensitivity, which in turn can be used to convert the acceleration
measurements for each axis into useful units. For a measurement $m$ with a full scale range $f$ the angular velocity in
degrees per second can be found like this:

$$
    \text{angular velocity} = m \cdot \frac{f}{2^{15}}
$$

\paragraph{*-Axis}
These fields represent the angular velocity measurements for each axis. These fields are signed integers in two's
complement format.

\subsubsection{GNSS Location}

The format of the GNSS location block is described in figure \ref{format:telem-gnss-location}.

\begin{figure}[h]
    \centering
    \begin{bytefield}[bitwidth=0.03\linewidth]{32}
        \bitheader{0-31} \\
        \wordbox{1}{Fix Time} \\
        \wordbox{1}{Latitude} \\
        \wordbox{1}{Longitude} \\
        \wordbox{1}{UTC Time} \\
        \wordbox{1}{Altitude} \\
        \bitbox{16}{Speed} & \bitbox{16}{Course} \\
        \bitbox{16}{PDOP} & \bitbox{16}{HDOP} \\
        \bitbox{16}{VDOP} & \bitbox{8}{Sats} & \bitbox{2}{Fix} &
        \bitbox{6}{\color{lightgray}\rule{\width}{\height}}

    \end{bytefield}
    \caption{GNSS Location Data Payload Format}
    \label{format:telem-gnss-location}
\end{figure}

\paragraph{Fix Time}
The mission time when the fix was received.

\paragraph{Latitude}
The latitude of the rocket in units of \SI{100}{\micro\arcminute\per LSB} (micro-arcminutes per least significant bit).
This field is a signed 32 bit integer in two's complement format.

\paragraph{Longitude}
The longitude of the rocket in units of \SI{100}{\micro\arcminute\per LSB} (micro-arcminutes per least significant
bit). This field is a signed 32 bit integer in two's complement format.

\paragraph{UTC Time}
The UTC time when the fix was received in seconds since the Unix epoch.

\paragraph{Altitude}
The altitude as calculated by the GNSS module in units of millimetres above sea level. This field is a signed 32 bit
integer in two's complement format.

\paragraph{Speed}
The speed over ground of the rocket in hundredths of a knot. This field is a signed 16 bit integer in two's complement
format.

\paragraph{Course}
The course over ground of the rocket in hundredths of a degree. This field is a signed 16 bit integer in two's
complement format.

\paragraph{PDOP}
Position Dilution of Precision * 100.

\paragraph{HDOP}
Horizontal Dilution of Precision * 100.

\paragraph{VDOP}
Vertical Dilution of Precision * 100.

\paragraph{Sats}
The number of GNSS satellites used in the fix.

\paragraph{Fix}
The fix type as shown in table \ref{table:gnss-fix-type}.

\begin{table*}[htb]
    \centering
    \begin{tabular}{@{}ll@{}}
        \toprule
        Fix Type Value & Description   \\
        \midrule
        0x0            & Unknown       \\
        0x1            & Not available \\
        0x2            & 2D fix        \\
        0x3            & 3D fix        \\
        \bottomrule
    \end{tabular}
    \caption{GNSS Fix Type Values}
    \label{table:gnss-fix-type}
\end{table*}

\subsubsection{GNSS Metadata}

The format of the GNSS location block is described in figure \ref{format:telem-gnss-metadata}. The length of this block
varies, after the GLONASS satellites in use bit field 4 bytes of information is encoded for each satellite in view of
the GNSS module. Note that not all satellites which are in view will also be in use.

\begin{figure}[h]
    \centering
    \begin{bytefield}[bitwidth=0.03\linewidth]{32}
        \bitheader{0-31} \\
        \wordbox{1}{Mission Time} \\
        \wordbox{1}{GPS Satellites in Use} \\
        \wordbox{1}{GLONASS Satellites in Use} \\
        \begin{rightwordgroup}{First Sat}
            \bitbox{8}{Elevation} & \bitbox{8}{SNR} & \bitbox{5}{ID} &
            \bitbox{9}{Azimuth} &
            \bitbox{1}{\color{lightgray}\rule{\width}{\height}} &
            \bitbox{1}{T}
        \end{rightwordgroup} \\
        \bitbox[]{32}{$\vdots$ \\[1ex]} \\
        \begin{rightwordgroup}{Last Sat}
            \bitbox{8}{Elevation} & \bitbox{8}{SNR} & \bitbox{5}{ID} &
            \bitbox{9}{Azimuth} &
            \bitbox{1}{\color{lightgray}\rule{\width}{\height}} &
            \bitbox{1}{T}
        \end{rightwordgroup}
    \end{bytefield}
    \caption{GNSS Metadata Payload Format}
    \label{format:telem-gnss-metadata}
\end{figure}

\paragraph{Mission Time}
The mission time when the fix was received.

\paragraph{GPS Satellites in Use}

This bit field indicates which GPS satellites are used in the current fix. Each bit position represents a GPS
pseudo-random noise sequence.

\paragraph{GLONASS Satellites in Use}
This bit field indicates which GLONASS satellites are used in the current fix. Each bit position represents a slot
number. Bit zero corresponds to slot 65, bit 1 to slot 66 and so on.

\paragraph{Elevation}
This field represents the elevation for a GNSS satellite in degrees.

\paragraph{SNR}
This field represents the signal to noise ratio for a signal from a satellite in dB Hz.

\paragraph{ID}
This field contains the satellite pseudo-ransom noise sequence for GPS satellites or the satellite ID for GLONASS
satellites.

\paragraph{Azimuth}
This field contains the satellite azimuth in degrees.

\paragraph{T}
This field encodes the satellite type, it is 0 for GPS satellites or 1 for GLONASS satellites.

\subsubsection{MPU9250 IMU Data}

The format of the MPU9250 IMU data block is described in figure \ref{format:telem-mpu9250-imu}. This block can contain
a variable amount of acceleration, angular velocity and magnetic flux density data. The format of each individual data
entry is described in figure \ref{format:telem-mpu9250-imu-entry}. Note that since each entry is 21 bytes long entries
after the first one may not be properly aligned.

\begin{figure}[h]
    \centering
    \begin{bytefield}[bitwidth=0.03\linewidth]{32}
        \bitheader{0-31} \\
        \wordbox{1}{Measurement Time} \\
        \bitbox{8}{A/G SR Div} & \bitbox{1}{\rotatebox{90}{\tiny MS}} &
        \bitbox{2}{\tiny A FSR} & \bitbox{2}{\tiny G FSR} &
        \bitbox{3}{A BW} & \bitbox{3}{G BW} &
        \bitbox{13}{\color{lightgray}\rule{\width}{\height}} &
        \wordbox[lr]{1}{IMU Data} \\
        \skippedwords \\
        \wordbox[lrb]{1}{} \\
    \end{bytefield}
    \caption{MPU9250 Inertial Measurement Unit Data Payload Format}
    \label{format:telem-mpu9250-imu}
\end{figure}

\paragraph{Measurement Time}
The time of the last measurement in the block.

\paragraph{A/G SR Div}
The accelerometer/gyroscope sample rate divisor field can be used to calculate the sample rate for the accelerometer
and gyroscope. The following equation can be used to find the sample rate:

$$
    \text{sample rate} = \frac{1000}{\text{A/G SR Div} + 1}
$$

\paragraph{MS}
This field encodes the sample rate for the magnetometer. The possible values for this field are shown in table
\ref{table:mpu9250-mag-odr}.

\begin{table*}[htb]
    \centering
    \begin{tabular}{@{}ll@{}}
        \toprule
        MS Value & Sample Rate        \\
        \midrule
        0x0      & {\SI{8}{\hertz}}   \\
        0x1      & {\SI{100}{\hertz}} \\
        \bottomrule
    \end{tabular}
    \caption{MPU9250 Magnetometer Sample Rate Values}
    \label{table:mpu9250-mag-odr}
\end{table*}

\paragraph{A FSR}
This field encodes the full scale range for the accelerometer. The possible values for this field are shown in table
\ref{table:mpu9250-a-fsr}.

\begin{table*}[htb]
    \centering
    \begin{tabular}{@{}ll@{}}
        \toprule
        Accel. FSR Value & Full Scale Range \\
        \midrule
        0x0              & {\SI{\pm 2}{g}}  \\
        0x1              & {\SI{\pm 4}{g}}  \\
        0x2              & {\SI{\pm 8}{g}}  \\
        0x3              & {\SI{\pm 16}{g}} \\
        \bottomrule
    \end{tabular}
    \caption{MPU9250 Accelerometer Full Scale Range Values}
    \label{table:mpu9250-a-fsr}
\end{table*}

\paragraph{G FSR}
This field encodes the full scale range for the gyroscope. The possible values for this field are shown in table
\ref{table:mpu9250-g-fsr}.

\begin{table*}[htb]
    \centering
    \begin{tabular}{@{}ll@{}}
        \toprule
        Gyro. FSR Value & Full Scale Range                    \\
        \midrule
        0x0             & {\SI{\pm 250}{\degree\per\second}}  \\
        0x1             & {\SI{\pm 500}{\degree\per\second}}  \\
        0x2             & {\SI{\pm 1000}{\degree\per\second}} \\
        0x3             & {\SI{\pm 2000}{\degree\per\second}} \\
        \bottomrule
    \end{tabular}
    \caption{MPU9250 Gyroscope Full Scale Range Values}
    \label{table:mpu9250-g-fsr}
\end{table*}

\paragraph{A BW}
This field encodes the accelerometer low pass filter bandwidth. The possible values for this field are shown in table
\ref{table:mpu9250-a-bw}.

\begin{table*}[htb]
    \centering
    \begin{tabular}{@{}ll@{}}
        \toprule
        Accel. BW Value & Low Pass Filter \SI{3}{\decibel} Bandwidth \\
        \midrule
        0x0             & {\SI{5.05}{\hertz}}                        \\
        0x1             & {\SI{10.2}{\hertz}}                        \\
        0x2             & {\SI{21.2}{\hertz}}                        \\
        0x3             & {\SI{44.8}{\hertz}}                        \\
        0x4             & {\SI{99}{\hertz}}                          \\
        0x5             & {\SI{218.1}{\hertz}}                       \\
        0x6             & {\SI{420}{\hertz}}                         \\
        \bottomrule
    \end{tabular}
    \caption{MPU9250 Accelerometer Low Pass Filter Bandwidth Values}
    \label{table:mpu9250-a-bw}
\end{table*}

\paragraph{G BW}
This field encodes the gyroscope low pass filter bandwidth. The possible values for this field are shown in table
\ref{table:mpu9250-g-bw}.

\begin{table*}[htb]
    \centering
    \begin{tabular}{@{}ll@{}}
        \toprule
        Gyro. BW Value & Low Pass Filter \SI{3}{\decibel} Bandwidth \\
        \midrule
        0x0            & {\SI{5}{\hertz}}                           \\
        0x1            & {\SI{10}{\hertz}}                          \\
        0x2            & {\SI{20}{\hertz}}                          \\
        0x3            & {\SI{41}{\hertz}}                          \\
        0x4            & {\SI{92}{\hertz}}                          \\
        0x5            & {\SI{184}{\hertz}}                         \\
        0x6            & {\SI{250}{\hertz}}                         \\
        \bottomrule
    \end{tabular}
    \caption{MPU9250 Gyroscope Low Pass Filter Bandwidth Values}
    \label{table:mpu9250-g-bw}
\end{table*}

\begin{figure}[h]
    \centering
    \begin{bytefield}[bitwidth=0.03\linewidth]{32}
        \bitheader{0-31} \\
        \bitbox{16}{Accel. X} & \bitbox{16}{Accel. Y} \\
        \bitbox{16}{Accel. Z} & \bitbox{16}{Temperature} \\
        \bitbox{16}{Gyro. X} & \bitbox{16}{Gyro. Y} \\
        \bitbox{16}{Gyro. Z} & \bitbox{16}{Mag. X} \\
        \bitbox{16}{Mag. Y} & \bitbox{16}{Mag. Z} \\
        \bitbox{3}{\color{lightgray}\rule{\width}{\height}} &
        \bitbox{1}{O} & \bitbox{1}{R} &
        \bitbox{3}{\color{lightgray}\rule{\width}{\height}}
    \end{bytefield}
    \caption{MPU9250 IMU Measurement Entry Format}
    \label{format:telem-mpu9250-imu-entry}
\end{figure}

\paragraph{Accel. *}
The acceleration data fields are big endian values encoded in two's complement.

\paragraph{Gyro. *}
The angular velocity data fields are big endian values encoded in two's complement.

\paragraph{Mag. *}
The magnetic flux density data fields are little endian values encoded in two's complement.

\paragraph{O}
This field indicates whether there was an overflow while measuring the magnetic flux density. If this bit is set the
magnetometer data for the sample is not valid.

\paragraph{R}
This field indicates the resolution of the magnetometer data. If this bit is set resolution is 16 bits, otherwise it is
14 bits.

\subsubsection{KX134-1211 Accelerometer Data}

The format of the KX134-1211 accelerometer data block is described in figure \ref{format:telem-kx134-accel}. This block
can contain a variable amount of acceleration data. This data will be in sets of three measurements (x, y, z), each
measurement is either 8 or 16 bits long and is encoded in two's complement format.

\begin{figure}[h]
    \centering
    \begin{bytefield}[bitwidth=0.03\linewidth]{32}
        \bitheader{0-31} \\
        \wordbox{1}{Measurement Time} \\
        \bitbox{4}{ODR} & \bitbox{2}{\tiny Range} &
        \bitbox{1}{\rotatebox{90}{\tiny Roll}} &
        \bitbox{1}{\rotatebox{90}{\tiny Res}} &
        \bitbox{6}{\color{lightgray}\rule{\width}{\height}} &
        \bitbox{2}{\small Pad} &
        \bitbox[lrt]{16}{} \\
        \wordbox[lr]{1}{Acceleration Data} \\
        \skippedwords \\
        \wordbox[lrb]{1}{} \\
    \end{bytefield}
    \caption{KX134-1211 Accelerometer Data Payload Format}
    \label{format:telem-kx134-accel}
\end{figure}

\paragraph{Measurement Time}
The time of the last measurement in the block.

\paragraph{ODR}
This field encodes the output data rate at which the accelerometer is configured. The possible values for this field
are shown in table \ref{table:kx134-odr}.

\begin{table*}[htb]
    \centering
    \begin{tabular}{@{}ll@{}}
        \toprule
        ODR Value & Output Data Rate     \\
        \midrule
        0x0       & {\SI{0.781}{\hertz}} \\
        0x1       & {\SI{1.563}{\hertz}} \\
        0x2       & {\SI{3.125}{\hertz}} \\
        0x3       & {\SI{6.25}{\hertz}}  \\
        0x4       & {\SI{12.5}{\hertz}}  \\
        0x5       & {\SI{25}{\hertz}}    \\
        0x6       & {\SI{50}{\hertz}}    \\
        0x7       & {\SI{100}{\hertz}}   \\
        0x8       & {\SI{200}{\hertz}}   \\
        0x9       & {\SI{400}{\hertz}}   \\
        0xa       & {\SI{800}{\hertz}}   \\
        0xb       & {\SI{1600}{\hertz}}  \\
        0xc       & {\SI{3200}{\hertz}}  \\
        0xd       & {\SI{6400}{\hertz}}  \\
        0xe       & {\SI{12800}{\hertz}} \\
        0xf       & {\SI{25600}{\hertz}} \\
        \bottomrule
    \end{tabular}
    \caption{KX134-1211 Output Data Rate Values}
    \label{table:kx134-odr}
\end{table*}

\paragraph{Range}
This field encodes the configured full scale range of the accelerometer. The possible values for this field are shown
in table \ref{table:kx134-fsr}.

\begin{table*}[htb]
    \centering
    \begin{tabular}{@{}ll@{}}
        \toprule
        Range Value & Full Scale Range \\
        \midrule
        0x0         & {\SI{\pm 8}{g}}  \\
        0x1         & {\SI{\pm 16}{g}} \\
        0x2         & {\SI{\pm 32}{g}} \\
        0x3         & {\SI{\pm 64}{g}} \\
        \bottomrule
    \end{tabular}
    \caption{KX134-1211 Full Scale Range Values}
    \label{table:kx134-fsr}
\end{table*}

\paragraph{Roll}
This field encode the accelerometer's configured low pass filter roll-off. The possible values for this field are shown
in table \ref{table:kx134-rolloff}.

\begin{table*}[htb]
    \centering
    \begin{tabular}{@{}ll@{}}
        \toprule
        Roll Value & Low Pass Filter Corner Frequency \\
        \midrule
        0x0        & $\text{ODR} / 9$                 \\
        0x1        & $\text{ODR} / 2$                 \\
        \bottomrule
    \end{tabular}
    \caption{KX134-1211 Low Pass Filter Rolloff Values}
    \label{table:kx134-rolloff}
\end{table*}

\paragraph{Res}
This field encodes the output resolution for the accelerometer. The possible values for this field are shown in table
\ref{table:kx134-res}.

\begin{table*}[htb]
    \centering
    \begin{tabular}{@{}ll@{}}
        \toprule
        Res Value & Resolution     \\
        \midrule
        0x0       & {\SI{8}{bit}}  \\
        0x1       & {\SI{16}{bit}} \\
        \bottomrule
    \end{tabular}
    \caption{KX134-1211 Resolution Values}
    \label{table:kx134-res}
\end{table*}

\paragraph{Pad}
This field indicates the number of padding bytes which present after the end of the acceleration data. This field is
important when parsing 8 bit data because it could appear that there is an extra sample due to padding.
