\subsection{Data Blocks}

Data blocks contain telemetry data. The possible sub-types data blocks are listed in table \ref{table:data-subtypes}.

Note that all \textbf{mission time} fields are measured in milliseconds since launch, and are all unsigned 32-bit
integers.

Any integer fields larger than 1 byte are in little-endian format. Any signed integers are in two's complement format.
This is because fields are packaged using C structs on a little endian machine (the de-facto standard).

\begin{table}[H]
    \centering
    \begin{tabular}{@{}ll@{}}
        \toprule
        Value             & Description             \\
        \midrule
        0x00              & Debug message           \\
        0x01              & Altitude                \\
        0x02              & Temperature             \\
        0x03              & Pressure                \\
        0x04              & Acceleration            \\
        0x05              & Angular velocity        \\
        0x06              & GNSS location           \\
        0x07              & GNSS metadata           \\
        0x08              & Humidity                \\
        0x09 through 0xFF & Reserved for future use \\
        \bottomrule
    \end{tabular}
    \caption{Data Block Subtypes}
    \label{table:data-subtypes}
\end{table}

% DATA BLOCK TYPES %
\subsubsection{Debug Message}

Debug messages are string messages intended to provide human readable debugging output. They are encoded as UTF-8
strings and do not require a terminating NULL character because the string length can be extrapolated from the block
length. Every debug message is preceded by a 32 bit unsigned mission time value.

Note that like all other data blocks debug messages must be a multiple of four bytes long. If the string to be logged
is not a multiple of four bytes it must be padded with NULL characters at the end to make it fit the required
alignment. The length recorded in the block header must be a multiple of four, and therefore must include any NULL
padding bytes at the end of the string.

\subsubsection{Altitude}

The altitude block is used to convey altimeter data. This block contains both the altitude value calculated by the
rocket and the raw pressure and temperature values if they are available. The format of the altitude block payload is
described in figure \ref{format:telem-altitude}.

\begin{figure}[h]
    \centering
    \begin{bytefield}[bitwidth=0.03\linewidth]{32}
        \bitheader{0-31} \\
        \wordbox{1}{Measurement Time} \\
        \wordbox{1}{Pressure} \\
        \wordbox{1}{Temperature} \\
        \wordbox{1}{Altitude}
    \end{bytefield}
    \caption{Altitude Data Payload Format}
    \label{format:telem-altitude}
\end{figure}

\paragraph{Measurement Time}
The mission time when the measurement was taken.

\paragraph{Pressure}
The measured pressure in units of Pascals. This field is a signed 32 bit integer in two's complement format.

\paragraph{Temperature}
The measured temperature in units of 1 millidegree Celsius/LSB. This field is a signed 32 bit integer in two's
complement format.

\paragraph{Altitude}
The calculated altitude in units of \SI{1}{\milli\meter\per LSB}. This field is a signed 32 bit integer in two's
complement format.


\subsubsection{Temperature}

The temperature data block reports the temperature detected from within the rocket in millidegrees Celsius. The format
of the temperature block payload is described in figure \ref{format:telem-temperature}.

\begin{figure}[H]
    \centering
    \begin{bytefield}[bitwidth=0.03\linewidth]{16}
        \bitheader{0-15} \\
        \wordbox{1}{Measurement Time} \\
        \wordbox{2}{Temperature} \\
    \end{bytefield}
    \caption{Temperature data block format}
    \label{format:telem-temperature}
\end{figure}

\blocktimestampexp

\paragraph{Temperature}

The measured temperature in units of millidegrees Celsius. This field is a signed 32 bit integer in two's complement
format.

\subsubsection{Pressure}

The pressure data block reports the pressure detected from within the rocket in Pascals. The format of the pressure
block payload is described in figure \ref{format:telem-pressure}.

\begin{figure}[h]
    \centering
    \begin{bytefield}[bitwidth=0.03\linewidth]{32}
        \bitheader{0-31} \\
        \wordbox{1}{Measurement Time} \\
        \wordbox{1}{Pressure} \\
    \end{bytefield}
    \caption{Pressure data block format}
    \label{format:telem-pressure}
\end{figure}

\paragraph{Measurement Time}
The mission time when the measurement was taken.

\paragraph{Pressure}
The measured pressure in units of Pascals. This field is an unsigned 32 bit integer.

\subsection{Linear Acceleration}

The linear acceleration block is used to convey generic 3-axis accelerometer data. This block is intended to abstract
over the details of any individual accelerometer. The format of the acceleration block payload is described in figure
\ref{format:telem-acceleration}.

\begin{figure}[H]
    \centering
    \begin{bytefield}[bitwidth=0.03\linewidth]{16}
        \bitheader{0-15} \\
        \wordbox{1}{Measurement Time} \\
        \wordbox{1}{X-Axis} \\
        \wordbox{1}{Y-Axis} \\
        \wordbox{1}{Z-Axis} \\
    \end{bytefield}
    \caption{Acceleration Data Payload Format}
    \label{format:telem-acceleration}
\end{figure}

\blocktimestampexp

\paragraph{*-Axis}

These fields represent the acceleration measurements for each axis. These fields are signed integers in two's
complement format.

The unit of measurement for each axis is measured in centimetres per second squared. That is, dividing the value in
each axis field by 100 will give the standard acceleration unit of metres per second squared.

The acceleration measurements are relative to the rocket's heading. If the IMU Z-axis is in parallel with the rocket,
then the acceleration in the z-axis is also a parallel vector.

\subsubsection{Angular Velocity}

The angular velocity block is used to convey generic 3-axis gyroscope data. This block is intended to abstract over the
details of any individual gyroscope. The format of the acceleration block payload is described in figure
\ref{format:telem-angular-velocity}.

\begin{figure}[h]
    \centering
    \begin{bytefield}[bitwidth=0.03\linewidth]{32}
        \bitheader{0-31} \\
        \wordbox{1}{Measurement Time} \\
        \bitbox{16}{Full Scale Range} & \bitbox{16}{X-Axis} \\
        \bitbox{16}{Y-Axis} & \bitbox{16}{Z-Axis}
    \end{bytefield}
    \caption{Angular Velocity Data Payload Format}
    \label{format:telem-angular-velocity}
\end{figure}

\paragraph{Measurement Time}
The mission time when the measurement was taken.

\paragraph{Full Scale Range}
The full scale range of the gyroscope in degrees per second. This value represents the maximum and minimum angular
velocity that can be measured, for example a value of 2000 in this field represents a full scale range of \SI{\pm
    2000}{degrees per second}. If the resolution of the gyroscope is less than 16 bits the full scale range must be
adjusted to compensate, for example a 12 bit gyroscope with a \SI{\pm 500}{degree per second} full scale range would
use the value 4000 for this field.

This value can be used to calculate the gyroscope sensitivity, which in turn can be used to convert the acceleration
measurements for each axis into useful units. For a measurement $m$ with a full scale range $f$ the angular velocity in
degrees per second can be found like this:

$$
    \text{angular velocity} = m \cdot \frac{f}{2^{15}}
$$

\paragraph{*-Axis}
These fields represent the angular velocity measurements for each axis. These fields are signed integers in two's
complement format.

\subsubsection{GNSS Location}

The format of the GNSS location block is described in figure \ref{format:telem-gnss-location}.

\begin{figure}[h]
    \centering
    \begin{bytefield}[bitwidth=0.03\linewidth]{32}
        \bitheader{0-31} \\
        \wordbox{1}{Fix Time} \\
        \wordbox{1}{Latitude} \\
        \wordbox{1}{Longitude} \\
        \wordbox{1}{UTC Time} \\
        \wordbox{1}{Altitude} \\
        \bitbox{16}{Speed} & \bitbox{16}{Course} \\
        \bitbox{16}{PDOP} & \bitbox{16}{HDOP} \\
        \bitbox{16}{VDOP} & \bitbox{8}{Sats} & \bitbox{2}{Fix} &
        \bitbox{6}{\color{lightgray}\rule{\width}{\height}}

    \end{bytefield}
    \caption{GNSS Location Data Payload Format}
    \label{format:telem-gnss-location}
\end{figure}

\paragraph{Fix Time}
The mission time when the fix was received.

\paragraph{Latitude}
The latitude of the rocket in units of \SI{100}{\micro\arcminute\per LSB} (micro-arcminutes per least significant bit).
This field is a signed 32 bit integer in two's complement format.

\paragraph{Longitude}
The longitude of the rocket in units of \SI{100}{\micro\arcminute\per LSB} (micro-arcminutes per least significant
bit). This field is a signed 32 bit integer in two's complement format.

\paragraph{UTC Time}
The UTC time when the fix was received in seconds since the Unix epoch.

\paragraph{Altitude}
The altitude as calculated by the GNSS module in units of millimetres above sea level. This field is a signed 32 bit
integer in two's complement format.

\paragraph{Speed}
The speed over ground of the rocket in hundredths of a knot. This field is a signed 16 bit integer in two's complement
format.

\paragraph{Course}
The course over ground of the rocket in hundredths of a degree. This field is a signed 16 bit integer in two's
complement format.

\paragraph{PDOP}
Position Dilution of Precision * 100.

\paragraph{HDOP}
Horizontal Dilution of Precision * 100.

\paragraph{VDOP}
Vertical Dilution of Precision * 100.

\paragraph{Sats}
The number of GNSS satellites used in the fix.

\paragraph{Fix}
The fix type as shown in table \ref{table:gnss-fix-type}.

\begin{table*}[htb]
    \centering
    \begin{tabular}{@{}ll@{}}
        \toprule
        Fix Type Value & Description   \\
        \midrule
        0x0            & Unknown       \\
        0x1            & Not available \\
        0x2            & 2D fix        \\
        0x3            & 3D fix        \\
        \bottomrule
    \end{tabular}
    \caption{GNSS Fix Type Values}
    \label{table:gnss-fix-type}
\end{table*}

\subsubsection{GNSS Metadata}

The format of the GNSS location block is described in figure \ref{format:telem-gnss-metadata}. The length of this block
varies, after the GLONASS satellites in use bit field 4 bytes of information is encoded for each satellite in view of
the GNSS module. Note that not all satellites which are in view will also be in use.

\begin{figure}[H]
    \centering
    \begin{bytefield}[bitwidth=0.03\linewidth]{32}
        \bitheader{0-31} \\
        \wordbox{1}{Mission Time} \\
        \wordbox{1}{GPS Satellites in Use} \\
        \wordbox{1}{GLONASS Satellites in Use} \\
        \begin{rightwordgroup}{First Sat}
            \bitbox{8}{Elevation} & \bitbox{8}{SNR} & \bitbox{5}{ID} &
            \bitbox{9}{Azimuth} &
            \bitbox{1}{\color{lightgray}\rule{\width}{\height}} &
            \bitbox{1}{T}
        \end{rightwordgroup} \\
        \bitbox[]{32}{$\vdots$ \\[1ex]} \\
        \begin{rightwordgroup}{Last Sat}
            \bitbox{8}{Elevation} & \bitbox{8}{SNR} & \bitbox{5}{ID} &
            \bitbox{9}{Azimuth} &
            \bitbox{1}{\color{lightgray}\rule{\width}{\height}} &
            \bitbox{1}{T}
        \end{rightwordgroup}
    \end{bytefield}
    \caption{GNSS Metadata Payload Format}
    \label{format:telem-gnss-metadata}
\end{figure}

\paragraph{Mission Time}
The mission time when the fix was received.

\paragraph{GPS Satellites in Use}

This bit field indicates which GPS satellites are used in the current fix. Each bit position represents a GPS
pseudo-random noise sequence.

\paragraph{GLONASS Satellites in Use}
This bit field indicates which GLONASS satellites are used in the current fix. Each bit position represents a slot
number. Bit zero corresponds to slot 65, bit 1 to slot 66 and so on.

\paragraph{Elevation}
This field represents the elevation for a GNSS satellite in degrees.

\paragraph{SNR}
This field represents the signal to noise ratio for a signal from a satellite in dB Hz.

\paragraph{ID}
This field contains the satellite pseudo-ransom noise sequence for GPS satellites or the satellite ID for GLONASS
satellites.

\paragraph{Azimuth}
This field contains the satellite azimuth in degrees.

\paragraph{T}
This field encodes the satellite type, it is 0 for GPS satellites or 1 for GLONASS satellites.


\subsubsection{Humidity}

The humidity block is used to convey the relative humidity inside of the rocket. The format of the humidity block is
described in figure \ref{format:telem-humidity}.

\begin{figure}[H]
    \centering
    \begin{bytefield}[bitwidth=0.03\linewidth]{16}
        \bitheader{0-15} \\
        \wordbox{1}{Measurement Time} \\
        \wordbox{2}{Humidity}
    \end{bytefield}
    \caption{Humidity Data Payload Format}
    \label{format:telem-humidity}
\end{figure}

\blocktimestampexp

\paragraph{Humidity}

The calculated relative humidity in ten thousandths of a percent (i.e 1\% is 100). This field is an unsigned 32 bit
integer.

