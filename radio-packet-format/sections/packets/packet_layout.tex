\section{Packet Layout}

CU InSpace's radio packets consist of a packet header followed by one or more segments called blocks. Each block has
its own header followed by a variable amount of data. All headers and data segments have a length that is a multiple of
four bytes. The layout of a packet is shown in figure \ref{format:packet}.

\begin{figure}[H]
    \centering
    \begin{bytefield}{16}
        \wordbox{2}{Packet Header} \\
        \begin{rightwordgroup}{Block 1}
            \wordbox{2}{Block Header} \\
            \wordbox[lrt]{1}{Block Contents} \\
            \skippedwords \\
            \wordbox[lrb]{1}{}
        \end{rightwordgroup} \\
        \wordbox[]{1}{$\vdots$} \\[1ex]
        \begin{rightwordgroup}{Block $N$}
            \wordbox{2}{Block Header} \\
            \wordbox[lrt]{1}{Block Contents} \\
            \skippedwords \\
            \wordbox[lrb]{1}{}
        \end{rightwordgroup} \\
    \end{bytefield}
    \caption{Layout of radio packet}
    \label{format:packet}
\end{figure}

\subsection{Packet Header}

Every packet is preceded by a 16 byte header. The header contains an amateur radio call sign and the packets total
length as well as other information which describes the packet as a whole. The packet header follows the format
described in figure \ref{format:packet-header}.

\begin{figure}[H]
    \centering
    \begin{bytefield}[bitwidth=0.03\linewidth]{32}
        \bitheader{0-31} \\
        \wordbox[lrt]{1}{Call Sign} \\ \bitbox[lr]{32}{} \\
        \bitbox[lrb]{8}{} & \bitbox{8}{Length} & \bitbox{8}{Version} & \bitbox{8}{Src. Addr.} \\
        \bitbox{32}{Packet Number} \\
    \end{bytefield}
    \caption{Packet header format}
    \label{format:packet-header}
\end{figure}

\paragraph{Call Sign}
The first 9 bytes of the packet must contain an amateur radio call sign in ASCII. If the call sign being used is less
than 9 bytes long, the call sign will be padded with NULL characters at the end.

The call sign field is 9 bytes long because this field is required to accommodate more than just a single Canadian call
sign of maximum 6 characters, but also Canadian call signs being used in the United States. Such call signs must be
suffixed with the call zone in which they are being used. \cite{foreign-broadcast}

For example, the Canadian amateur call sign of \texttt{VA3INI} would become \texttt{VA3INI/W5}, where \texttt{W5} is
the 5th call zone (coincidentally the call zone for the location where Spaceport America Cup is hosted).

There are 10 call zones in the United States, represented using digits 0-9. \cite{us-callzones} The digit is preceded
by a forward slash and a capital 'W', for a total of three additional characters on top of the original 6 for a
Canadian HAM radio call sign.

\paragraph{Length}
The length field indicates the total length of the packet in bytes. This includes the packet header itself.

The length of a packet must always be a multiple of four bytes. Because of this the length field is encoded in units of
four bytes per least significant bit. Because a packet with a length of zero is not possible, the value in the length
field is one less than the length of the packet divided by four. This results in a possible range of packet lengths of
4 to 1024 bytes in multiples of four (though a valid packet must have a length of at least 16 bytes).

The following equations show how the length of a packet in bytes can be converted to the corresponding length value and
vice versa.

\begin{align*}
    \text{Length}                 & = \frac{\text{packet length in bytes}}{4} - 1 \\
    \text{packet length in bytes} & = \left(\text{Length} + 1\right) \cdot 4
\end{align*}

\paragraph{Version}
The \emph{version} field indicates what version of the radio packet format the packet adheres to. The receiver should
not attempt to parse a packet that has a version number which it does not recognize. The \emph{call sign} and
\emph{version} fields will remain for all future versions of this packet format, all other fields could change.

\paragraph{Source Address}
This field indicates the source of the packet. Possible device addresses are listed in table \ref{table:dev-addresses}.
0xF is not a valid source address.

\begin{table}[H]
    \centering
    \begin{tabular}{@{}ll@{}}
        \toprule
        Address           & Device                  \\
        \midrule
        0x00              & Ground Station          \\
        0x01              & Rocket                  \\
        0x02 through 0xFE & Reserved for future use \\
        0xFF              & Multicast               \\
        \bottomrule
    \end{tabular}
    \caption{Device addresses}
    \label{table:dev-addresses}
\end{table}

\paragraph{Packet Number}
This field contains a count value that should be incremented with every packet transmission from a given device.

\subsection{Block Header}
Every block starts with a four byte block header. This header indicates the type of information contained in the block
as well as the blocks’s intended recipient. The block header is formatted as described in Figure
\ref{format:block-header}.

\begin{figure}[H]
    \centering
    \begin{bytefield}[bitwidth=0.03\linewidth]{32}
        \bitheader{0-31} \\
        \bitbox{8}{Block Length} &
        \bitbox{8}{Type} &
        \bitbox{8}{Subtype} &
        \bitbox{8}{\small Dest. Addr.}
    \end{bytefield}
    \caption{Block header format}
    \label{format:block-header}
\end{figure}

\paragraph{Block Length}
The block length indicates how many bytes are in the block, including the block header. The offset of the block after
the current block is equal to the sum of the offset of the current block and the current blocks length.

The length of a block must always be a multiple of four, therefore the block length field is encoded in units of four
bytes per least significant bit. Because a valid block will always contain at least 4 bytes, the value of the block
Length field is one less than the length of the block divided by four. This means that the field has a range from four
bytes to 128 bytes in four byte increments.

The following equations show how the length of a block in bytes can be converted to the corresponding length value and
vice versa.

\begin{align*}
    \text{Block Length}          & = \frac{\text{block length in bytes}}{4} - 1   \\
    \text{block length in bytes} & = \left(\text{Block Length} + 1\right) \cdot 4
\end{align*}

If the length of a block is such that some or all of the block would exceed the total packet length, then the block is
invalid and will be ignored.

\paragraph{Type}
The block type field indicates the purpose of the block. Possible values are listed in table \ref{table:block-types}.
Block types are discussed in more detail in section \ref{sec:block-formats}.

\begin{table}[H]
    \centering
    \begin{tabular}{@{}ll@{}}
        \toprule
        Value             & Block Type              \\
        \midrule
        0x00              & Data                    \\
        0x01 through 0xFF & Reserved for future use \\
        \bottomrule
    \end{tabular}
    \caption{Block types}
    \label{table:block-types}
\end{table}

\paragraph{Sub-type}
The sub-type indicates the type of information or request contained in the block. The possible block sub-types
for each block type are listed and discussed in more detail in section \ref{sec:block-formats}.

\paragraph{Destination Address}
This field indicates what device the block is intended for. Received blocks which are intended for a different device
should be ignored. Possible addresses are listed in table \ref{table:dev-addresses}. A destination address of
\lstinline{0xFF} indicates that the block is intended for anyone who receives it. \lstinline{0xFF} indicates that the
block is intended for anyone who receives it.
