\subsection{Altitude}

The altitude block is used to convey the altitude of the rocket. The format of the altitude block payload is described
in Figure \ref{format:telem-altitude}.

\begin{figure}[H]
    \centering
    \begin{bytefield}[bitwidth=0.03\linewidth]{16}
        \bitheader{0-15} \\
        \wordbox{1}{Measurement Time} \\
        \wordbox{2}{Altitude} \\
    \end{bytefield}
    \caption{Altitude Data Payload Format}
    \label{format:telem-altitude}
\end{figure}

\blocktimestampexp

\paragraph{Altitude}

The calculated altitude in units of millimetres. This field is a signed 32 bit integer in two's complement format.

If the block header specifies the type for altitude above sea level, this measurement should be interpreted as
millimetres above sea level.

If the block header specifies the type for altitude above launch height (ground level), this measurement should be
interpreted as millimetres from the starting height of the rocket.

See Table \ref{table:block-types} for the different altitude types.
